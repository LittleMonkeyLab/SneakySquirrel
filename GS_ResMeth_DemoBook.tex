% Options for packages loaded elsewhere
\PassOptionsToPackage{unicode}{hyperref}
\PassOptionsToPackage{hyphens}{url}
\PassOptionsToPackage{dvipsnames,svgnames,x11names}{xcolor}
%
\documentclass[
  letterpaper,
  DIV=11,
  numbers=noendperiod]{scrreprt}

\usepackage{amsmath,amssymb}
\usepackage{iftex}
\ifPDFTeX
  \usepackage[T1]{fontenc}
  \usepackage[utf8]{inputenc}
  \usepackage{textcomp} % provide euro and other symbols
\else % if luatex or xetex
  \usepackage{unicode-math}
  \defaultfontfeatures{Scale=MatchLowercase}
  \defaultfontfeatures[\rmfamily]{Ligatures=TeX,Scale=1}
\fi
\usepackage{lmodern}
\ifPDFTeX\else  
    % xetex/luatex font selection
    \setmainfont[]{Atkinson Hyperlegible}
\fi
% Use upquote if available, for straight quotes in verbatim environments
\IfFileExists{upquote.sty}{\usepackage{upquote}}{}
\IfFileExists{microtype.sty}{% use microtype if available
  \usepackage[]{microtype}
  \UseMicrotypeSet[protrusion]{basicmath} % disable protrusion for tt fonts
}{}
\makeatletter
\@ifundefined{KOMAClassName}{% if non-KOMA class
  \IfFileExists{parskip.sty}{%
    \usepackage{parskip}
  }{% else
    \setlength{\parindent}{0pt}
    \setlength{\parskip}{6pt plus 2pt minus 1pt}}
}{% if KOMA class
  \KOMAoptions{parskip=half}}
\makeatother
\usepackage{xcolor}
\setlength{\emergencystretch}{3em} % prevent overfull lines
\setcounter{secnumdepth}{5}
% Make \paragraph and \subparagraph free-standing
\makeatletter
\ifx\paragraph\undefined\else
  \let\oldparagraph\paragraph
  \renewcommand{\paragraph}{
    \@ifstar
      \xxxParagraphStar
      \xxxParagraphNoStar
  }
  \newcommand{\xxxParagraphStar}[1]{\oldparagraph*{#1}\mbox{}}
  \newcommand{\xxxParagraphNoStar}[1]{\oldparagraph{#1}\mbox{}}
\fi
\ifx\subparagraph\undefined\else
  \let\oldsubparagraph\subparagraph
  \renewcommand{\subparagraph}{
    \@ifstar
      \xxxSubParagraphStar
      \xxxSubParagraphNoStar
  }
  \newcommand{\xxxSubParagraphStar}[1]{\oldsubparagraph*{#1}\mbox{}}
  \newcommand{\xxxSubParagraphNoStar}[1]{\oldsubparagraph{#1}\mbox{}}
\fi
\makeatother

\usepackage{color}
\usepackage{fancyvrb}
\newcommand{\VerbBar}{|}
\newcommand{\VERB}{\Verb[commandchars=\\\{\}]}
\DefineVerbatimEnvironment{Highlighting}{Verbatim}{commandchars=\\\{\}}
% Add ',fontsize=\small' for more characters per line
\usepackage{framed}
\definecolor{shadecolor}{RGB}{241,243,245}
\newenvironment{Shaded}{\begin{snugshade}}{\end{snugshade}}
\newcommand{\AlertTok}[1]{\textcolor[rgb]{0.68,0.00,0.00}{#1}}
\newcommand{\AnnotationTok}[1]{\textcolor[rgb]{0.37,0.37,0.37}{#1}}
\newcommand{\AttributeTok}[1]{\textcolor[rgb]{0.40,0.45,0.13}{#1}}
\newcommand{\BaseNTok}[1]{\textcolor[rgb]{0.68,0.00,0.00}{#1}}
\newcommand{\BuiltInTok}[1]{\textcolor[rgb]{0.00,0.23,0.31}{#1}}
\newcommand{\CharTok}[1]{\textcolor[rgb]{0.13,0.47,0.30}{#1}}
\newcommand{\CommentTok}[1]{\textcolor[rgb]{0.37,0.37,0.37}{#1}}
\newcommand{\CommentVarTok}[1]{\textcolor[rgb]{0.37,0.37,0.37}{\textit{#1}}}
\newcommand{\ConstantTok}[1]{\textcolor[rgb]{0.56,0.35,0.01}{#1}}
\newcommand{\ControlFlowTok}[1]{\textcolor[rgb]{0.00,0.23,0.31}{\textbf{#1}}}
\newcommand{\DataTypeTok}[1]{\textcolor[rgb]{0.68,0.00,0.00}{#1}}
\newcommand{\DecValTok}[1]{\textcolor[rgb]{0.68,0.00,0.00}{#1}}
\newcommand{\DocumentationTok}[1]{\textcolor[rgb]{0.37,0.37,0.37}{\textit{#1}}}
\newcommand{\ErrorTok}[1]{\textcolor[rgb]{0.68,0.00,0.00}{#1}}
\newcommand{\ExtensionTok}[1]{\textcolor[rgb]{0.00,0.23,0.31}{#1}}
\newcommand{\FloatTok}[1]{\textcolor[rgb]{0.68,0.00,0.00}{#1}}
\newcommand{\FunctionTok}[1]{\textcolor[rgb]{0.28,0.35,0.67}{#1}}
\newcommand{\ImportTok}[1]{\textcolor[rgb]{0.00,0.46,0.62}{#1}}
\newcommand{\InformationTok}[1]{\textcolor[rgb]{0.37,0.37,0.37}{#1}}
\newcommand{\KeywordTok}[1]{\textcolor[rgb]{0.00,0.23,0.31}{\textbf{#1}}}
\newcommand{\NormalTok}[1]{\textcolor[rgb]{0.00,0.23,0.31}{#1}}
\newcommand{\OperatorTok}[1]{\textcolor[rgb]{0.37,0.37,0.37}{#1}}
\newcommand{\OtherTok}[1]{\textcolor[rgb]{0.00,0.23,0.31}{#1}}
\newcommand{\PreprocessorTok}[1]{\textcolor[rgb]{0.68,0.00,0.00}{#1}}
\newcommand{\RegionMarkerTok}[1]{\textcolor[rgb]{0.00,0.23,0.31}{#1}}
\newcommand{\SpecialCharTok}[1]{\textcolor[rgb]{0.37,0.37,0.37}{#1}}
\newcommand{\SpecialStringTok}[1]{\textcolor[rgb]{0.13,0.47,0.30}{#1}}
\newcommand{\StringTok}[1]{\textcolor[rgb]{0.13,0.47,0.30}{#1}}
\newcommand{\VariableTok}[1]{\textcolor[rgb]{0.07,0.07,0.07}{#1}}
\newcommand{\VerbatimStringTok}[1]{\textcolor[rgb]{0.13,0.47,0.30}{#1}}
\newcommand{\WarningTok}[1]{\textcolor[rgb]{0.37,0.37,0.37}{\textit{#1}}}

\providecommand{\tightlist}{%
  \setlength{\itemsep}{0pt}\setlength{\parskip}{0pt}}\usepackage{longtable,booktabs,array}
\usepackage{calc} % for calculating minipage widths
% Correct order of tables after \paragraph or \subparagraph
\usepackage{etoolbox}
\makeatletter
\patchcmd\longtable{\par}{\if@noskipsec\mbox{}\fi\par}{}{}
\makeatother
% Allow footnotes in longtable head/foot
\IfFileExists{footnotehyper.sty}{\usepackage{footnotehyper}}{\usepackage{footnote}}
\makesavenoteenv{longtable}
\usepackage{graphicx}
\makeatletter
\newsavebox\pandoc@box
\newcommand*\pandocbounded[1]{% scales image to fit in text height/width
  \sbox\pandoc@box{#1}%
  \Gscale@div\@tempa{\textheight}{\dimexpr\ht\pandoc@box+\dp\pandoc@box\relax}%
  \Gscale@div\@tempb{\linewidth}{\wd\pandoc@box}%
  \ifdim\@tempb\p@<\@tempa\p@\let\@tempa\@tempb\fi% select the smaller of both
  \ifdim\@tempa\p@<\p@\scalebox{\@tempa}{\usebox\pandoc@box}%
  \else\usebox{\pandoc@box}%
  \fi%
}
% Set default figure placement to htbp
\def\fps@figure{htbp}
\makeatother
% definitions for citeproc citations
\NewDocumentCommand\citeproctext{}{}
\NewDocumentCommand\citeproc{mm}{%
  \begingroup\def\citeproctext{#2}\cite{#1}\endgroup}
\makeatletter
 % allow citations to break across lines
 \let\@cite@ofmt\@firstofone
 % avoid brackets around text for \cite:
 \def\@biblabel#1{}
 \def\@cite#1#2{{#1\if@tempswa , #2\fi}}
\makeatother
\newlength{\cslhangindent}
\setlength{\cslhangindent}{1.5em}
\newlength{\csllabelwidth}
\setlength{\csllabelwidth}{3em}
\newenvironment{CSLReferences}[2] % #1 hanging-indent, #2 entry-spacing
 {\begin{list}{}{%
  \setlength{\itemindent}{0pt}
  \setlength{\leftmargin}{0pt}
  \setlength{\parsep}{0pt}
  % turn on hanging indent if param 1 is 1
  \ifodd #1
   \setlength{\leftmargin}{\cslhangindent}
   \setlength{\itemindent}{-1\cslhangindent}
  \fi
  % set entry spacing
  \setlength{\itemsep}{#2\baselineskip}}}
 {\end{list}}
\usepackage{calc}
\newcommand{\CSLBlock}[1]{\hfill\break\parbox[t]{\linewidth}{\strut\ignorespaces#1\strut}}
\newcommand{\CSLLeftMargin}[1]{\parbox[t]{\csllabelwidth}{\strut#1\strut}}
\newcommand{\CSLRightInline}[1]{\parbox[t]{\linewidth - \csllabelwidth}{\strut#1\strut}}
\newcommand{\CSLIndent}[1]{\hspace{\cslhangindent}#1}

\usepackage{tcolorbox}
\tcbuselibrary{listings, breakable, skins}
\usepackage{enumitem}
\usepackage{fontawesome5}
\usepackage{fontspec}

% Set Atkinson Hyperlegible as main font
\setmainfont{Atkinson Hyperlegible}

\tcbset{
  gsbox/.style={
    enhanced,
    sharp corners,
    breakable,
    colback=white,
    colframe=black,
    fonttitle=\bfseries,
    before skip=10pt,
    after skip=10pt,
    boxrule=0.8pt,
    left=10pt,
    right=10pt,
    top=5pt,
    bottom=5pt
  }
}

\newtcolorbox{gslearningoutcome}[1][]{gsbox, title=\faBook\quad Learning Outcomes, colframe=orange!70!black, colback=orange!5, #1}
\newtcolorbox{gsactivity}[1][]{gsbox, title=\faBolt\quad Activity, colframe=cyan!70!black, colback=cyan!5, #1}
\newtcolorbox{gswebexercise}[1][]{gsbox, title=\faKeyboard\quad Web Exercise, colframe=teal!70!black, colback=teal!5, #1}
\newtcolorbox{gsglossary}[1][]{gsbox, title=\faBookOpen\quad Glossary, colframe=yellow!70!black, colback=yellow!10, #1}
\KOMAoption{captions}{tableheading}
\makeatletter
\@ifpackageloaded{tcolorbox}{}{\usepackage[skins,breakable]{tcolorbox}}
\@ifpackageloaded{fontawesome5}{}{\usepackage{fontawesome5}}
\definecolor{quarto-callout-color}{HTML}{909090}
\definecolor{quarto-callout-note-color}{HTML}{0758E5}
\definecolor{quarto-callout-important-color}{HTML}{CC1914}
\definecolor{quarto-callout-warning-color}{HTML}{EB9113}
\definecolor{quarto-callout-tip-color}{HTML}{00A047}
\definecolor{quarto-callout-caution-color}{HTML}{FC5300}
\definecolor{quarto-callout-color-frame}{HTML}{acacac}
\definecolor{quarto-callout-note-color-frame}{HTML}{4582ec}
\definecolor{quarto-callout-important-color-frame}{HTML}{d9534f}
\definecolor{quarto-callout-warning-color-frame}{HTML}{f0ad4e}
\definecolor{quarto-callout-tip-color-frame}{HTML}{02b875}
\definecolor{quarto-callout-caution-color-frame}{HTML}{fd7e14}
\makeatother
\makeatletter
\@ifpackageloaded{bookmark}{}{\usepackage{bookmark}}
\makeatother
\makeatletter
\@ifpackageloaded{caption}{}{\usepackage{caption}}
\AtBeginDocument{%
\ifdefined\contentsname
  \renewcommand*\contentsname{Table of contents}
\else
  \newcommand\contentsname{Table of contents}
\fi
\ifdefined\listfigurename
  \renewcommand*\listfigurename{List of Figures}
\else
  \newcommand\listfigurename{List of Figures}
\fi
\ifdefined\listtablename
  \renewcommand*\listtablename{List of Tables}
\else
  \newcommand\listtablename{List of Tables}
\fi
\ifdefined\figurename
  \renewcommand*\figurename{Figure}
\else
  \newcommand\figurename{Figure}
\fi
\ifdefined\tablename
  \renewcommand*\tablename{Table}
\else
  \newcommand\tablename{Table}
\fi
}
\@ifpackageloaded{float}{}{\usepackage{float}}
\floatstyle{ruled}
\@ifundefined{c@chapter}{\newfloat{codelisting}{h}{lop}}{\newfloat{codelisting}{h}{lop}[chapter]}
\floatname{codelisting}{Listing}
\newcommand*\listoflistings{\listof{codelisting}{List of Listings}}
\makeatother
\makeatletter
\makeatother
\makeatletter
\@ifpackageloaded{caption}{}{\usepackage{caption}}
\@ifpackageloaded{subcaption}{}{\usepackage{subcaption}}
\makeatother
\makeatletter
\@ifpackageloaded{fontawesome5}{}{\usepackage{fontawesome5}}
\makeatother

\usepackage{bookmark}

\IfFileExists{xurl.sty}{\usepackage{xurl}}{} % add URL line breaks if available
\urlstyle{same} % disable monospaced font for URLs
\hypersetup{
  pdftitle={GS\_ResMeth\_DemoBook},
  pdfauthor={Dr.~Gordon Wright and the gang},
  colorlinks=true,
  linkcolor={blue},
  filecolor={Maroon},
  citecolor={Blue},
  urlcolor={Blue},
  pdfcreator={LaTeX via pandoc}}


\title{GS\_ResMeth\_DemoBook}
\author{Dr.~Gordon Wright and the gang}
\date{2025-04-29}

\begin{document}
\maketitle

\renewcommand*\contentsname{Table of contents}
{
\hypersetup{linkcolor=}
\setcounter{tocdepth}{2}
\tableofcontents
}

\bookmarksetup{startatroot}

\chapter*{Preface}\label{preface}
\addcontentsline{toc}{chapter}{Preface}

\markboth{Preface}{Preface}

This is a Quarto book.

\faIcon{thumbs-up}

\faIcon{folder}

\faIcon{chess-pawn}

\faIcon{bluetooth}

\begin{gslearningoutcome}

\begin{itemize}
\tightlist
\item
  Understand partial and semipartial correlation
\item
  Use multiple regression with interpretation
\end{itemize}

\end{gslearningoutcome}

More content

\begin{gsactivity}

\textbf{Activity}

Explore the relationship between anxiety and reaction time using
regression analysis.

\end{gsactivity}

More content

\begin{gslearningoutcome}

\textbf{Learning Outcomes}

\begin{itemize}
\tightlist
\item
  Understand partial and semipartial correlation\\
\item
  Use multiple regression with interpretation
\end{itemize}

\end{gslearningoutcome}

More content

\begin{gswebexercise}

\textbf{Web Exercise}

Use the \href{https://webr.r-wasm.org/}{WebR console} to rerun the
regression with a different outcome variable.

\end{gswebexercise}

More content

\begin{gsglossary}

\textbf{Glossary}

\textbf{Multicollinearity}: A statistical phenomenon where two or more
predictors in a model are highly correlated.

\end{gsglossary}

say (\textbf{Student1908?})

\bookmarksetup{startatroot}

\chapter{Introduction}\label{introduction}

This is a book created from markdown and executable code.

See Knuth (1984) for additional discussion of literate programming.

\bookmarksetup{startatroot}

\chapter{Demonstrations}\label{demonstrations}

Demonstrations of stuff

\hfill\break

Here is a demo of web-r \href{webr.qmd}{web-r demo}

Here is a demo of webexercises \href{webexercises.qmd}{webexercises
demo}

Here is a demo of a download button \href{downloadthis.qmd}{downloadthis
demo}

Here is a demo of a Countdown timer

\section{another}\label{another}

\bookmarksetup{startatroot}

\chapter{Summary}\label{summary}

In summary, this book has no content whatsoever.

\bookmarksetup{startatroot}

\chapter{Webexercises}\label{webexercises}

This is a Web Exercise template created by the
\href{http://www.psy.gla.ac.uk}{psychology teaching team at the
University of Glasgow}, based on ideas from
\href{https://software-carpentry.org/lessons/}{Software Carpentry}. This
template shows how instructors can easily create interactive web
documents that students can use in self-guided learning.

The \texttt{\{webexercises\}} package provides a number of functions
that you use in
\href{https://github.com/rstudio/cheatsheets/raw/master/rmarkdown-2.0.pdf}{inline
R code} or through code chunk options to create HTML widgets (text
boxes, pull down menus, buttons that reveal hidden content). Examples
are given below. Render this file to HTML to see how it works.

\textbf{NOTE: To use the widgets in the compiled HTML file, you need to
have a JavaScript-enabled browser.}

\section{Example Questions}\label{example-questions}

\subsection{\texorpdfstring{Fill-In-The-Blanks
(\texttt{fitb()})}{Fill-In-The-Blanks (fitb())}}\label{fill-in-the-blanks-fitb}

Create fill-in-the-blank questions using \texttt{fitb()}, providing the
answer as the first argument.

\begin{itemize}
\tightlist
\item
  2 + 2 is \_
\end{itemize}

You can also create these questions dynamically, using variables from
your R session.

\begin{itemize}
\tightlist
\item
  The square root of 25 is: \_
\end{itemize}

The blanks are case-sensitive; if you don't care about case, use the
argument \texttt{ignore\_case\ =\ TRUE}.

\begin{itemize}
\tightlist
\item
  What is the letter after D? \_
\end{itemize}

If you want to ignore differences in whitespace use, use the argument
\texttt{ignore\_ws\ =\ TRUE} (which is the default) and include spaces
in your answer anywhere they could be acceptable.

\begin{itemize}
\tightlist
\item
  How do you load the tidyverse package?
  \_\_\_\_\_\_\_\_\_\_\_\_\_\_\_\_\_\_\_\_
\end{itemize}

You can set more than one possible correct answer by setting the answers
as a vector.

\begin{itemize}
\tightlist
\item
  Type a vowel: \_
\end{itemize}

You can use regular expressions to test answers against more complex
rules.

\begin{itemize}
\tightlist
\item
  Type any 3 letters: \_\_\_
\end{itemize}

\subsection{\texorpdfstring{Multiple Choice
(\texttt{mcq()})}{Multiple Choice (mcq())}}\label{multiple-choice-mcq}

\begin{itemize}
\item
  ``Never gonna give you up, never gonna:
\item
  \begin{enumerate}
  \def\labelenumi{(\Alph{enumi})}
  \tightlist
  \item
    let you go\\
  \end{enumerate}
\item
  \begin{enumerate}
  \def\labelenumi{(\Alph{enumi})}
  \setcounter{enumi}{1}
  \tightlist
  \item
    turn you down\\
  \end{enumerate}
\item
  \begin{enumerate}
  \def\labelenumi{(\Alph{enumi})}
  \setcounter{enumi}{2}
  \tightlist
  \item
    run away\\
  \end{enumerate}
\item
  \begin{enumerate}
  \def\labelenumi{(\Alph{enumi})}
  \setcounter{enumi}{3}
  \tightlist
  \item
    let you down
  \end{enumerate}
\end{itemize}

'' - ``I

\begin{itemize}
\item
  \begin{enumerate}
  \def\labelenumi{(\Alph{enumi})}
  \tightlist
  \item
    bless the rains\\
  \end{enumerate}
\item
  \begin{enumerate}
  \def\labelenumi{(\Alph{enumi})}
  \setcounter{enumi}{1}
  \tightlist
  \item
    guess it rains\\
  \end{enumerate}
\item
  \begin{enumerate}
  \def\labelenumi{(\Alph{enumi})}
  \setcounter{enumi}{2}
  \tightlist
  \item
    sense the rain
  \end{enumerate}

  down in Africa'' -Toto
\end{itemize}

\subsection{\texorpdfstring{True or False
(\texttt{torf()})}{True or False (torf())}}\label{true-or-false-torf}

\begin{itemize}
\tightlist
\item
  True or False? You can permute values in a vector using
  \texttt{sample()}. TRUE / FALSE
\end{itemize}

\subsection{\texorpdfstring{Longer MCQs
(\texttt{longmcq()})}{Longer MCQs (longmcq())}}\label{longer-mcqs-longmcq}

When your answers are very long, sometimes a drop-down select box gets
formatted oddly. You can use \texttt{longmcq()} to deal with this. Since
the answers are long, It's probably best to set up the options inside an
R chunk with \texttt{echo=FALSE}.

\textbf{What is a p-value?}

\begin{itemize}
\tightlist
\item
  \begin{enumerate}
  \def\labelenumi{(\Alph{enumi})}
  \tightlist
  \item
    the probability that the null hypothesis is true\\
  \end{enumerate}
\item
  \begin{enumerate}
  \def\labelenumi{(\Alph{enumi})}
  \setcounter{enumi}{1}
  \tightlist
  \item
    the probability of the observed, or more extreme, data, under the
    assumption that the null-hypothesis is true\\
  \end{enumerate}
\item
  \begin{enumerate}
  \def\labelenumi{(\Alph{enumi})}
  \setcounter{enumi}{2}
  \tightlist
  \item
    the probability of making an error in your conclusion
  \end{enumerate}
\end{itemize}

\textbf{What is true about a 95\% confidence interval of the mean?}

\begin{itemize}
\tightlist
\item
  \begin{enumerate}
  \def\labelenumi{(\Alph{enumi})}
  \tightlist
  \item
    95\% of the data fall within this range\\
  \end{enumerate}
\item
  \begin{enumerate}
  \def\labelenumi{(\Alph{enumi})}
  \setcounter{enumi}{1}
  \tightlist
  \item
    if you repeated the process many times, 95\% of intervals calculated
    in this way contain the true mean\\
  \end{enumerate}
\item
  \begin{enumerate}
  \def\labelenumi{(\Alph{enumi})}
  \setcounter{enumi}{2}
  \tightlist
  \item
    there is a 95\% probability that the true mean lies within this
    range
  \end{enumerate}
\end{itemize}

\section{Checked sections}\label{checked-sections}

Create sections with the class \texttt{webex-check} to add a button that
hides feedback until it is pressed. Add the class \texttt{webex-box} to
draw a box around the section (or use your own styles).

I am going to learn a lot: TRUE / FALSE

What is a p-value?

\begin{itemize}
\tightlist
\item
  \begin{enumerate}
  \def\labelenumi{(\Alph{enumi})}
  \tightlist
  \item
    the probability that the null hypothesis is true\\
  \end{enumerate}
\item
  \begin{enumerate}
  \def\labelenumi{(\Alph{enumi})}
  \setcounter{enumi}{1}
  \tightlist
  \item
    the probability of the observed, or more extreme, data, under the
    assumption that the null-hypothesis is true\\
  \end{enumerate}
\item
  \begin{enumerate}
  \def\labelenumi{(\Alph{enumi})}
  \setcounter{enumi}{2}
  \tightlist
  \item
    the probability of making an error in your conclusion
  \end{enumerate}
\end{itemize}

\section{Hidden solutions and hints}\label{hidden-solutions-and-hints}

You can fence off a solution area that will be hidden behind a button
using \texttt{hide()} before the solution and \texttt{unhide()} after,
each as inline R code. Pass the text you want to appear on the button to
the \texttt{hide()} function.

If the solution is a code chunk, instead of using \texttt{hide()} and
\texttt{unhide()}, simply set the \texttt{webex.hide} chunk option to
TRUE, or set it to the string you wish to display on the button.

\textbf{Recreate the scatterplot below, using the built-in \texttt{cars}
dataset.}

\pandocbounded{\includegraphics[keepaspectratio]{webexercises_files/figure-pdf/unnamed-chunk-6-1.pdf}}

I need a hint

See the documentation for \texttt{plot()} (\texttt{?plot})

Click here to see the solution

\begin{Shaded}
\begin{Highlighting}[]
\FunctionTok{plot}\NormalTok{(cars}\SpecialCharTok{$}\NormalTok{speed, cars}\SpecialCharTok{$}\NormalTok{dist)}
\end{Highlighting}
\end{Shaded}

\section{Deception Example Quiz}\label{deception-example-quiz}

This set of exercises will test your knowledge about various aspects of
deception in psychology. Answer the questions to check your
understanding of key concepts, theories, and research findings in this
area.

\section{Fill-In-The-Blanks}\label{fill-in-the-blanks}

\begin{enumerate}
\def\labelenumi{\arabic{enumi}.}
\item
  The tendency for people to believe they are less likely to be deceived
  than others is known as the
  \_\_\_\_\_\_\_\_\_\_\_\_\_\_\_\_\_\_\_\_\_\_\_\_\_\_\_ to deception.
\item
  In deception research, the \_\_\_\_\_\_\_\_\_\_\_ is an actor who
  works with the experimenter to deceive the actual participant.
\item
  The \_\_\_\_\_\_\_\_\_\_\_\_\_\_ is a genuine smile that involves both
  the mouth and the eyes, making it harder to fake in deceptive
  situations.
\end{enumerate}

\section{Multiple Choice}\label{multiple-choice}

\begin{enumerate}
\def\labelenumi{\arabic{enumi}.}
\tightlist
\item
  Which of the following is NOT typically considered a reliable
  indicator of deception?
\end{enumerate}

\begin{itemize}
\tightlist
\item
  \begin{enumerate}
  \def\labelenumi{(\Alph{enumi})}
  \tightlist
  \item
    Increased blinking\\
  \end{enumerate}
\item
  \begin{enumerate}
  \def\labelenumi{(\Alph{enumi})}
  \setcounter{enumi}{1}
  \tightlist
  \item
    Reduced hand gestures\\
  \end{enumerate}
\item
  \begin{enumerate}
  \def\labelenumi{(\Alph{enumi})}
  \setcounter{enumi}{2}
  \tightlist
  \item
    Lack of eye contact\\
  \end{enumerate}
\item
  \begin{enumerate}
  \def\labelenumi{(\Alph{enumi})}
  \setcounter{enumi}{3}
  \tightlist
  \item
    Increased speech errors
  \end{enumerate}
\end{itemize}

\begin{enumerate}
\def\labelenumi{\arabic{enumi}.}
\setcounter{enumi}{1}
\tightlist
\item
  The theory that proposes that lying is more cognitively demanding than
  telling the truth is called:
\end{enumerate}

\begin{itemize}
\tightlist
\item
  \begin{enumerate}
  \def\labelenumi{(\Alph{enumi})}
  \tightlist
  \item
    Interpersonal Deception Theory\\
  \end{enumerate}
\item
  \begin{enumerate}
  \def\labelenumi{(\Alph{enumi})}
  \setcounter{enumi}{1}
  \tightlist
  \item
    Cognitive Load Theory\\
  \end{enumerate}
\item
  \begin{enumerate}
  \def\labelenumi{(\Alph{enumi})}
  \setcounter{enumi}{2}
  \tightlist
  \item
    Four-Factor Theory\\
  \end{enumerate}
\item
  \begin{enumerate}
  \def\labelenumi{(\Alph{enumi})}
  \setcounter{enumi}{3}
  \tightlist
  \item
    Self-Presentation Theory
  \end{enumerate}
\end{itemize}

\begin{enumerate}
\def\labelenumi{\arabic{enumi}.}
\setcounter{enumi}{2}
\tightlist
\item
  In a typical deception study, who is usually unaware of the true
  nature of the experiment?
\end{enumerate}

\begin{itemize}
\tightlist
\item
  \begin{enumerate}
  \def\labelenumi{(\Alph{enumi})}
  \tightlist
  \item
    Participant\\
  \end{enumerate}
\item
  \begin{enumerate}
  \def\labelenumi{(\Alph{enumi})}
  \setcounter{enumi}{1}
  \tightlist
  \item
    Experimenter\\
  \end{enumerate}
\item
  \begin{enumerate}
  \def\labelenumi{(\Alph{enumi})}
  \setcounter{enumi}{2}
  \tightlist
  \item
    Confederate\\
  \end{enumerate}
\item
  \begin{enumerate}
  \def\labelenumi{(\Alph{enumi})}
  \setcounter{enumi}{3}
  \tightlist
  \item
    Research Assistant
  \end{enumerate}
\end{itemize}

\section{True or False}\label{true-or-false}

\begin{enumerate}
\def\labelenumi{\arabic{enumi}.}
\item
  Polygraph tests are highly accurate and widely accepted in scientific
  communities as reliable lie detectors. TRUE / FALSE
\item
  People are generally better at detecting lies told by strangers than
  by those close to them. TRUE / FALSE
\item
  Microexpressions are brief, involuntary facial expressions that can
  potentially reveal concealed emotions in deceptive situations. TRUE /
  FALSE
\end{enumerate}

\section{Longer MCQs}\label{longer-mcqs}

\textbf{Which of the following best describes the ``Truth-Default
Theory'' in deception research?}

\begin{itemize}
\tightlist
\item
  \begin{enumerate}
  \def\labelenumi{(\Alph{enumi})}
  \tightlist
  \item
    The theory that humans are naturally inclined to always tell the
    truth\\
  \end{enumerate}
\item
  \begin{enumerate}
  \def\labelenumi{(\Alph{enumi})}
  \setcounter{enumi}{1}
  \tightlist
  \item
    The theory that humans tend to believe others are telling the truth
    unless given reason to think otherwise\\
  \end{enumerate}
\item
  \begin{enumerate}
  \def\labelenumi{(\Alph{enumi})}
  \setcounter{enumi}{2}
  \tightlist
  \item
    The theory that truth-telling is easier and requires less cognitive
    effort than lying\\
  \end{enumerate}
\item
  \begin{enumerate}
  \def\labelenumi{(\Alph{enumi})}
  \setcounter{enumi}{3}
  \tightlist
  \item
    The theory that cultural norms universally prioritize truthfulness
    over deception
  \end{enumerate}
\end{itemize}

\textbf{What is the primary focus of Interpersonal Deception Theory
(IDT)?}

\begin{itemize}
\tightlist
\item
  \begin{enumerate}
  \def\labelenumi{(\Alph{enumi})}
  \tightlist
  \item
    The dynamic interaction between the deceiver and the target of
    deception\\
  \end{enumerate}
\item
  \begin{enumerate}
  \def\labelenumi{(\Alph{enumi})}
  \setcounter{enumi}{1}
  \tightlist
  \item
    The cultural variations in perceptions and acceptance of deceptive
    practices\\
  \end{enumerate}
\item
  \begin{enumerate}
  \def\labelenumi{(\Alph{enumi})}
  \setcounter{enumi}{2}
  \tightlist
  \item
    The psychological motivations that drive individuals to engage in
    deceptive behavior\\
  \end{enumerate}
\item
  \begin{enumerate}
  \def\labelenumi{(\Alph{enumi})}
  \setcounter{enumi}{3}
  \tightlist
  \item
    The neurological processes involved in creating and maintaining a
    lie
  \end{enumerate}
\end{itemize}

\section{Checked sections}\label{checked-sections-1}

Research suggests that detecting deception is:

\begin{itemize}
\tightlist
\item
  \begin{enumerate}
  \def\labelenumi{(\Alph{enumi})}
  \tightlist
  \item
    Easy for most people\\
  \end{enumerate}
\item
  \begin{enumerate}
  \def\labelenumi{(\Alph{enumi})}
  \setcounter{enumi}{1}
  \tightlist
  \item
    Only possible with specialized training\\
  \end{enumerate}
\item
  \begin{enumerate}
  \def\labelenumi{(\Alph{enumi})}
  \setcounter{enumi}{2}
  \tightlist
  \item
    Difficult, with accuracy rates often only slightly above chance\\
  \end{enumerate}
\item
  \begin{enumerate}
  \def\labelenumi{(\Alph{enumi})}
  \setcounter{enumi}{3}
  \tightlist
  \item
    Impossible without technological aids
  \end{enumerate}
\end{itemize}

Explain why detecting deception is challenging for most people:

Click for explanation

Detecting deception is challenging for several reasons:

\begin{enumerate}
\def\labelenumi{\arabic{enumi}.}
\tightlist
\item
  Many common beliefs about deception cues (e.g., lack of eye contact)
  are not reliable indicators.
\item
  Liars often strategically control their behavior to appear truthful.
\item
  Individual differences in baseline behavior make it hard to identify
  deviations.
\item
  The cognitive load of trying to detect lies can impair judgment.
\item
  Confirmation bias can lead people to interpret ambiguous cues in line
  with their expectations.
\end{enumerate}

\section{Hidden solutions and hints}\label{hidden-solutions-and-hints-1}

\textbf{A researcher wants to study how cognitive load affects lying
behavior. Describe a potential experimental design to investigate this.}

I need a hint

Consider how you might manipulate cognitive load (e.g., through a
secondary task) and measure lying behavior. Think about what control
conditions you might need.

Click here to see a possible experimental design

\begin{Shaded}
\begin{Highlighting}[]
\CommentTok{\# Possible Experimental Design:}

\CommentTok{\# Participants: 100 adults randomly assigned to two groups (50 each)}
\CommentTok{\# }
\CommentTok{\# Procedure:}
\CommentTok{\# 1. All participants are asked to lie about a recent experience}
\CommentTok{\# }
\CommentTok{\# 2. Experimental group: }
\CommentTok{\#    {-} Must count backwards from 100 by 7s while lying (high cognitive load)}
\CommentTok{\# }
\CommentTok{\# 3. Control group:}
\CommentTok{\#    {-} Simply lie without additional task (normal cognitive load)}
\CommentTok{\# }
\CommentTok{\# 4. Measure dependent variables:}
\CommentTok{\#    {-} Speech hesitations}
\CommentTok{\#    {-} Speech rate}
\CommentTok{\#    {-} Amount of detail provided}
\CommentTok{\#    {-} Perceived believability (rated by independent judges)}
\CommentTok{\# }
\CommentTok{\# 5. Compare measures between groups to assess the effect of cognitive load on lying behavior}

\CommentTok{\# This design allows us to isolate the effect of increased cognitive load on various aspects of lying behavior.}
\end{Highlighting}
\end{Shaded}

This set of exercises covers various aspects of deception psychology,
including theories, research methods, and key findings. It utilizes
different question types to engage learners and test their understanding
of the subject matter.

\bookmarksetup{startatroot}

\chapter{webR in Quarto HTML
Documents}\label{webr-in-quarto-html-documents}

Each class session has an interactive lesson that you will work through
\textbf{\emph{after}} doing the readings and watching the lecture. These
lessons are a central part of the class---they will teach you how to use
\{ggplot2\} and other packages in the tidyverse to create beautiful and
truthful visualizations with R.

Interactive code sections look like this. Make changes in the text box
and click on the green ``Run Code'' button to see the results. Sometimes
there will be a tab with a hint or solution.

\begin{tcolorbox}[enhanced jigsaw, colframe=quarto-callout-important-color-frame, leftrule=.75mm, breakable, toprule=.15mm, arc=.35mm, rightrule=.15mm, colback=white, opacityback=0, bottomrule=.15mm, left=2mm]
\begin{minipage}[t]{5.5mm}
\textcolor{quarto-callout-important-color}{\faExclamation}
\end{minipage}%
\begin{minipage}[t]{\textwidth - 5.5mm}

\vspace{-3mm}\textbf{Your turn}\vspace{3mm}

Modify the code here to show the relationship between health and wealth
for 2002 instead of 2007.

\section{\texorpdfstring{\faIcon{code} Interactive
editor}{ Interactive editor}}

\begin{Shaded}
\begin{Highlighting}[]
\NormalTok{gapminder\_filtered \textless{}{-} gapminder |\textgreater{}}
\NormalTok{  filter(year == 2007)}

\NormalTok{ggplot(data = gapminder\_filtered,}
\NormalTok{       mapping = aes(x = gdpPercap, y = lifeExp, }
\NormalTok{                     size = pop, color = continent)) +}
\NormalTok{  geom\_point() + }
\NormalTok{  scale\_x\_log10(labels = scales::dollar\_format(accuracy = 1)) +}
\NormalTok{  scale\_size\_continuous(labels = scales::label\_comma()) +}
\NormalTok{  scale\_color\_viridis\_d(option = "plasma") +}
\NormalTok{  labs(x = "GDP per capita", y = "Life expectancy",}
\NormalTok{       title = "Health and wealth are strongly related",}
\NormalTok{       subtitle = "142 countries; 2007 only", caption = "Source: The Gapminder Project",}
\NormalTok{       color = "Continent", size = "Population") +}
\NormalTok{  theme\_bw()}
\end{Highlighting}
\end{Shaded}

\section{\texorpdfstring{\faIcon{lightbulb} Hint}{ Hint}}

\textbf{Hint:} You'll want to change something in the code that creates
\texttt{gapminder\_filtered}. The text in the subtitle won't change
automatically, so you'll want to edit that too.

\end{minipage}%
\end{tcolorbox}

If you're curious how this works, each interactive code section uses
\href{https://quarto-webr.thecoatlessprofessor.com/}{the amazing
\{quarto-webr\} package} to run R directly in your browser.

\bookmarksetup{startatroot}

\chapter{DownloadThis Demo}\label{downloadthis-demo}

How to download files and stuff

\hfill\break

\begin{Shaded}
\begin{Highlighting}[]
\FunctionTok{library}\NormalTok{(downloadthis)}
\end{Highlighting}
\end{Shaded}

\section{What a terrific Website}\label{what-a-terrific-website}

and this is some sparkling content.

How could I possibly download the data you mention?

There you go, mydude!

and

\begin{tcolorbox}[enhanced jigsaw, colframe=quarto-callout-tip-color-frame, leftrule=.75mm, breakable, toprule=.15mm, arc=.35mm, rightrule=.15mm, colback=white, opacityback=0, bottomrule=.15mm, left=2mm]
\begin{minipage}[t]{5.5mm}
\textcolor{quarto-callout-tip-color}{\faLightbulb}
\end{minipage}%
\begin{minipage}[t]{\textwidth - 5.5mm}

In the foundations of inference chapters, we have provided three
different methods for statistical inference. We will continue to build
on all three of the methods throughout the text, and by the end, you
should have an understanding of the similarities and differences between
them. Meanwhile, it is important to note that the methods are designed
to mimic variability with data, and we know that variability can come
from different sources (e.g., random sampling vs.~random allocation, see
\textbf{?@fig-randsampValloc}). In \textbf{?@tbl-foundations-summary},
we have summarized some of the ways the inferential procedures feature
specific sources of variability. We hope that you refer back to the
table often as you dive more deeply into inferential ideas in future
chapters.

\end{minipage}%
\end{tcolorbox}

\bookmarksetup{startatroot}

\chapter*{References}\label{references}
\addcontentsline{toc}{chapter}{References}

\markboth{References}{References}

\phantomsection\label{refs}
\begin{CSLReferences}{1}{0}
\bibitem[\citeproctext]{ref-knuth84}
Knuth, Donald E. 1984. \emph{The TeXbook}. Addison-Wesley.

\end{CSLReferences}

\part{Overview}

\chapter*{Preface}\label{preface-1}
\addcontentsline{toc}{chapter}{Preface}

\markboth{Preface}{Preface}

This is a Quarto book.

\faIcon{thumbs-up}

\faIcon{folder}

\faIcon{chess-pawn}

\faIcon{bluetooth}

\begin{gslearningoutcome}

\begin{itemize}
\tightlist
\item
  Understand partial and semipartial correlation
\item
  Use multiple regression with interpretation
\end{itemize}

\end{gslearningoutcome}

More content

\begin{gsactivity}

\textbf{Activity}

Explore the relationship between anxiety and reaction time using
regression analysis.

\end{gsactivity}

More content

\begin{gslearningoutcome}

\textbf{Learning Outcomes}

\begin{itemize}
\tightlist
\item
  Understand partial and semipartial correlation\\
\item
  Use multiple regression with interpretation
\end{itemize}

\end{gslearningoutcome}

More content

\begin{gswebexercise}

\textbf{Web Exercise}

Use the \href{https://webr.r-wasm.org/}{WebR console} to rerun the
regression with a different outcome variable.

\end{gswebexercise}

More content

\begin{gsglossary}

\textbf{Glossary}

\textbf{Multicollinearity}: A statistical phenomenon where two or more
predictors in a model are highly correlated.

\end{gsglossary}

say (\textbf{Student1908?})

\part{Week 01}

\chapter{Lecture}\label{lecture}

\chapter{PS53011C/PS71020E Lab
Worksheet}\label{ps53011cps71020e-lab-worksheet}

\subsection{📊 Lab Instructions}

This week's lab will explore multiple regression using tidyverse
principles!

\begin{center}\rule{0.5\linewidth}{0.5pt}\end{center}

\subsection{📽️ Slides}

\begin{quote}
🧭 \textbf{These are your slides for the week!}

Use the arrow keys to navigate through the slides:

\begin{itemize}
\tightlist
\item
  ⬅️ and ➡️ to move left and right
\item
  ⬆️ and ⬇️ for nested (vertical) slides
\item
  Press \texttt{Esc} to view the \textbf{slide overview}
\item
  Press \texttt{F} for \textbf{fullscreen} (or use browser controls)
\item
  Press \texttt{O} to toggle the \textbf{overview mode}
\item
  Press \texttt{S} to open the \textbf{Speaker Notes} window (in
  presenter mode)
\end{itemize}

💡 Slides are best viewed in \textbf{fullscreen} for clarity.
\end{quote}

\section{Week 4: Multiple Regression (Part
1)}\label{week-4-multiple-regression-part-1}

\begin{quote}
Please attempt all questions in your own words. Model answers will be
available on the VLE page following the lab session.
\end{quote}

\begin{center}\rule{0.5\linewidth}{0.5pt}\end{center}

\phantomsection\label{sec-lab01-LO}
\section{Learning Outcomes}\label{learning-outcomes}

\begin{enumerate}
\def\labelenumi{\arabic{enumi}.}
\tightlist
\item
  Conduct simple and multiple linear regression analyses using R and the
  Tidyverse.\\
\item
  Explore relationships between regression coefficients and correlation
  measures.\\
\item
  Assess assumptions of linear regression including normality,
  linearity, and multicollinearity.\\
\item
  Compute and interpret simple, partial, and semipartial correlations in
  R.
\end{enumerate}

\begin{center}\rule{0.5\linewidth}{0.5pt}\end{center}

\section{Materials}\label{materials}

\begin{itemize}
\tightlist
\item
  \textbf{Software}: R (Tidyverse package)
\item
  \textbf{Dataset}:
\end{itemize}

\begin{Shaded}
\begin{Highlighting}[]
\FunctionTok{library}\NormalTok{(tidyverse)}
\end{Highlighting}
\end{Shaded}

\begin{center}\rule{0.5\linewidth}{0.5pt}\end{center}

\section{Dataset Overview}\label{dataset-overview}

The dataset includes reaction time (RT) data for participants responding
to emotional facial expressions. Of particular interest is the average
RT to fearful faces for correct identifications. Predictors include:

\begin{itemize}
\tightlist
\item
  \texttt{traitanx}: Trait anxiety (Spielberger scale)
\item
  \texttt{ACS}: Attentional Control Scale score
\item
  \texttt{Age}: Participant age
\end{itemize}

\begin{center}\rule{0.5\linewidth}{0.5pt}\end{center}

\section{Task 1: Descriptive Statistics and
Visualisation}\label{task-1-descriptive-statistics-and-visualisation}

\begin{itemize}
\tightlist
\item
  Load the dataset
\item
  Create summary statistics and visualizations for RT, traitanx, ACS,
  and Age.
\end{itemize}

\begin{Shaded}
\begin{Highlighting}[]
\NormalTok{fearful\_data }\OtherTok{\textless{}{-}} \FunctionTok{read\_csv}\NormalTok{(}\StringTok{"data/fearful\_faces.csv"}\NormalTok{)}

\NormalTok{fearful\_data }\SpecialCharTok{\%\textgreater{}\%}
  \FunctionTok{summarise}\NormalTok{(}\FunctionTok{across}\NormalTok{(}\FunctionTok{c}\NormalTok{(Fearful\_face\_RT, Happy\_face\_RT, traitanx, ACS, Age), }\FunctionTok{list}\NormalTok{(}\AttributeTok{mean =}\NormalTok{ mean, }\AttributeTok{sd =}\NormalTok{ sd), }\AttributeTok{na.rm =} \ConstantTok{TRUE}\NormalTok{))}
\end{Highlighting}
\end{Shaded}

\begin{verbatim}
# A tibble: 1 x 10
  Fearful_face_RT_mean Fearful_face_RT_sd Happy_face_RT_mean Happy_face_RT_sd
                 <dbl>              <dbl>              <dbl>            <dbl>
1                0.491             0.0604              0.489           0.0600
# i 6 more variables: traitanx_mean <dbl>, traitanx_sd <dbl>, ACS_mean <dbl>,
#   ACS_sd <dbl>, Age_mean <dbl>, Age_sd <dbl>
\end{verbatim}

\begin{Shaded}
\begin{Highlighting}[]
\NormalTok{fearful\_data }\SpecialCharTok{\%\textgreater{}\%}
  \FunctionTok{pivot\_longer}\NormalTok{(}\AttributeTok{cols =} \FunctionTok{c}\NormalTok{(Fearful\_face\_RT, Happy\_face\_RT, traitanx, ACS, Age), }\AttributeTok{names\_to =} \StringTok{"variable"}\NormalTok{, }\AttributeTok{values\_to =} \StringTok{"value"}\NormalTok{) }\SpecialCharTok{\%\textgreater{}\%}
  \FunctionTok{ggplot}\NormalTok{(}\FunctionTok{aes}\NormalTok{(}\AttributeTok{x =}\NormalTok{ value)) }\SpecialCharTok{+}
  \FunctionTok{geom\_histogram}\NormalTok{(}\AttributeTok{bins =} \DecValTok{30}\NormalTok{, }\AttributeTok{fill =} \StringTok{"steelblue"}\NormalTok{, }\AttributeTok{color =} \StringTok{"white"}\NormalTok{) }\SpecialCharTok{+}
  \FunctionTok{facet\_wrap}\NormalTok{(}\SpecialCharTok{\textasciitilde{}}\NormalTok{ variable, }\AttributeTok{scales =} \StringTok{"free"}\NormalTok{) }\SpecialCharTok{+}
  \FunctionTok{theme\_minimal}\NormalTok{()}
\end{Highlighting}
\end{Shaded}

\pandocbounded{\includegraphics[keepaspectratio]{week01/lab_files/figure-pdf/descriptives-1.pdf}}

\begin{center}\rule{0.5\linewidth}{0.5pt}\end{center}

\section{Task 2: Correlation
Analysis}\label{task-2-correlation-analysis}

\begin{itemize}
\tightlist
\item
  Compute a correlation matrix
\item
  Visualize relationships using scatterplots
\end{itemize}

\begin{Shaded}
\begin{Highlighting}[]
\NormalTok{fearful\_data }\SpecialCharTok{\%\textgreater{}\%}
  \FunctionTok{select}\NormalTok{(Fearful\_face\_RT, Happy\_face\_RT, traitanx, ACS, Age) }\SpecialCharTok{\%\textgreater{}\%}
  \FunctionTok{cor}\NormalTok{(}\AttributeTok{use =} \StringTok{"complete.obs"}\NormalTok{) }\SpecialCharTok{\%\textgreater{}\%}
  \FunctionTok{round}\NormalTok{(}\DecValTok{2}\NormalTok{)}
\end{Highlighting}
\end{Shaded}

\begin{verbatim}
                Fearful_face_RT Happy_face_RT traitanx   ACS   Age
Fearful_face_RT            1.00          0.96    -0.33  0.18  0.04
Happy_face_RT              0.96          1.00    -0.37  0.14  0.02
traitanx                  -0.33         -0.37     1.00 -0.36 -0.14
ACS                        0.18          0.14    -0.36  1.00  0.36
Age                        0.04          0.02    -0.14  0.36  1.00
\end{verbatim}

\begin{Shaded}
\begin{Highlighting}[]
\FunctionTok{pairs}\NormalTok{(fearful\_data }\SpecialCharTok{\%\textgreater{}\%} \FunctionTok{select}\NormalTok{(Fearful\_face\_RT, Happy\_face\_RT, traitanx, ACS, Age), }\AttributeTok{main =} \StringTok{"Scatterplot Matrix"}\NormalTok{)}
\end{Highlighting}
\end{Shaded}

\pandocbounded{\includegraphics[keepaspectratio]{week01/lab_files/figure-pdf/correlations-1.pdf}}

\begin{center}\rule{0.5\linewidth}{0.5pt}\end{center}

\section{Task 3: Multiple Regression}\label{task-3-multiple-regression}

\begin{itemize}
\tightlist
\item
  Conduct a multiple regression predicting Fearful\_face\_RT from
  traitanx, ACS, and Age.
\end{itemize}

\begin{Shaded}
\begin{Highlighting}[]
\NormalTok{model }\OtherTok{\textless{}{-}} \FunctionTok{lm}\NormalTok{(Fearful\_face\_RT }\SpecialCharTok{\textasciitilde{}}\NormalTok{ traitanx }\SpecialCharTok{+}\NormalTok{ ACS }\SpecialCharTok{+}\NormalTok{ Age, }\AttributeTok{data =}\NormalTok{ fearful\_data)}
\FunctionTok{summary}\NormalTok{(model)}
\end{Highlighting}
\end{Shaded}

\begin{verbatim}

Call:
lm(formula = Fearful_face_RT ~ traitanx + ACS + Age, data = fearful_data)

Residuals:
      Min        1Q    Median        3Q       Max 
-0.105742 -0.043660 -0.002261  0.044587  0.116670 

Coefficients:
              Estimate Std. Error t value Pr(>|t|)    
(Intercept)  0.5470081  0.0637205   8.584  1.9e-12 ***
traitanx    -0.0018929  0.0007492  -2.527   0.0138 *  
ACS          0.0005506  0.0009446   0.583   0.5619    
Age         -0.0002299  0.0010605  -0.217   0.8290    
---
Signif. codes:  0 '***' 0.001 '**' 0.01 '*' 0.05 '.' 0.1 ' ' 1

Residual standard error: 0.05698 on 68 degrees of freedom
  (17931 observations deleted due to missingness)
Multiple R-squared:  0.1149,    Adjusted R-squared:  0.07587 
F-statistic: 2.943 on 3 and 68 DF,  p-value: 0.0391
\end{verbatim}

\begin{center}\rule{0.5\linewidth}{0.5pt}\end{center}

\section{Task 4: Model Diagnostics}\label{task-4-model-diagnostics}

\begin{itemize}
\tightlist
\item
  Check linear regression assumptions.
\end{itemize}

\begin{Shaded}
\begin{Highlighting}[]
\FunctionTok{par}\NormalTok{(}\AttributeTok{mfrow =} \FunctionTok{c}\NormalTok{(}\DecValTok{2}\NormalTok{, }\DecValTok{2}\NormalTok{))}
\FunctionTok{plot}\NormalTok{(model)}
\end{Highlighting}
\end{Shaded}

\pandocbounded{\includegraphics[keepaspectratio]{week01/lab_files/figure-pdf/diagnostics-1.pdf}}

\begin{center}\rule{0.5\linewidth}{0.5pt}\end{center}

\section{Task 5: Partial and Semipartial
Correlations}\label{task-5-partial-and-semipartial-correlations}

\begin{Shaded}
\begin{Highlighting}[]
\CommentTok{\# Load required libraries}
\FunctionTok{library}\NormalTok{(tidyverse)}
\FunctionTok{library}\NormalTok{(ppcor)}


\CommentTok{\# Calculate partial correlations}
\NormalTok{partial\_data }\OtherTok{\textless{}{-}}\NormalTok{ fearful\_data }\SpecialCharTok{\%\textgreater{}\%}
\NormalTok{  dplyr}\SpecialCharTok{::}\FunctionTok{select}\NormalTok{(Fearful\_face\_RT, traitanx, ACS, Age) }\SpecialCharTok{\%\textgreater{}\%}
  \FunctionTok{drop\_na}\NormalTok{()}
\NormalTok{pcor\_result }\OtherTok{\textless{}{-}} \FunctionTok{pcor}\NormalTok{(partial\_data, }\AttributeTok{method =} \StringTok{"pearson"}\NormalTok{)}
\NormalTok{pcor\_result}
\end{Highlighting}
\end{Shaded}

\begin{verbatim}
$estimate
                Fearful_face_RT    traitanx         ACS         Age
Fearful_face_RT      1.00000000 -0.29295465  0.07051172 -0.02627832
traitanx            -0.29295465  1.00000000 -0.29776049 -0.02300362
ACS                  0.07051172 -0.29776049  1.00000000  0.33458467
Age                 -0.02627832 -0.02300362  0.33458467  1.00000000

$p.value
                Fearful_face_RT   traitanx         ACS         Age
Fearful_face_RT      0.00000000 0.01384887 0.561885102 0.829035513
traitanx             0.01384887 0.00000000 0.012298804 0.850075844
ACS                  0.56188510 0.01229880 0.000000000 0.004640019
Age                  0.82903551 0.85007584 0.004640019 0.000000000

$statistic
                Fearful_face_RT   traitanx        ACS        Age
Fearful_face_RT       0.0000000 -2.5266180  0.5829054 -0.2167714
traitanx             -2.5266180  0.0000000 -2.5720632 -0.1897429
ACS                   0.5829054 -2.5720632  0.0000000  2.9277975
Age                  -0.2167714 -0.1897429  2.9277975  0.0000000

$n
[1] 72

$gp
[1] 2

$method
[1] "pearson"
\end{verbatim}

\begin{center}\rule{0.5\linewidth}{0.5pt}\end{center}

\section{Task 6: Semipartial (Part)
Correlation}\label{task-6-semipartial-part-correlation}

\begin{quote}
This task explores how to isolate the unique contribution of one
predictor (e.g., trait anxiety) to a dependent variable (reaction time),
controlling for other variables only in the predictor.
\end{quote}

\subsection{Objective}\label{objective}

\begin{itemize}
\tightlist
\item
  Calculate a semipartial correlation between \texttt{Fearful\_face\_RT}
  and \texttt{traitanx}, controlling for \texttt{ACS} and \texttt{Age}
  \textbf{only in the predictor}.
\end{itemize}

\begin{Shaded}
\begin{Highlighting}[]
\CommentTok{\# Load tidyverse if not already}
\FunctionTok{library}\NormalTok{(tidyverse)}

\CommentTok{\# Ensure your data is clean}
\NormalTok{semipartial\_data }\OtherTok{\textless{}{-}}\NormalTok{ fearful\_data }\SpecialCharTok{\%\textgreater{}\%}
\NormalTok{  dplyr}\SpecialCharTok{::}\FunctionTok{select}\NormalTok{(Fearful\_face\_RT, traitanx, ACS, Age) }\SpecialCharTok{\%\textgreater{}\%}
  \FunctionTok{drop\_na}\NormalTok{()}

\CommentTok{\# Step 1: Residualize the predictor (traitanx \textasciitilde{} ACS + Age)}
\NormalTok{resid\_traitanx }\OtherTok{\textless{}{-}} \FunctionTok{lm}\NormalTok{(traitanx }\SpecialCharTok{\textasciitilde{}}\NormalTok{ ACS }\SpecialCharTok{+}\NormalTok{ Age, }\AttributeTok{data =}\NormalTok{ semipartial\_data)}\SpecialCharTok{$}\NormalTok{residuals}

\CommentTok{\# Step 2: Compute correlation between raw DV and residualized predictor}
\NormalTok{semipartial\_corr }\OtherTok{\textless{}{-}} \FunctionTok{cor}\NormalTok{(semipartial\_data}\SpecialCharTok{$}\NormalTok{Fearful\_face\_RT, resid\_traitanx)}
\NormalTok{semipartial\_corr}
\end{Highlighting}
\end{Shaded}

\begin{verbatim}
[1] -0.2882545
\end{verbatim}

\begin{tcolorbox}[enhanced jigsaw, colframe=quarto-callout-note-color-frame, leftrule=.75mm, breakable, toprule=.15mm, arc=.35mm, rightrule=.15mm, colback=white, opacityback=0, bottomrule=.15mm, left=2mm]
\begin{minipage}[t]{5.5mm}
\textcolor{quarto-callout-note-color}{\faInfo}
\end{minipage}%
\begin{minipage}[t]{\textwidth - 5.5mm}

\textbf{Interpretation:} This semipartial correlation represents the
unique association between trait anxiety and fearful face reaction time,
controlling for ACS and Age \textbf{only in trait anxiety}. Unlike
partial correlation, it leaves the DV unadjusted.

\end{minipage}%
\end{tcolorbox}

\begin{center}\rule{0.5\linewidth}{0.5pt}\end{center}

\section{Reflection}\label{reflection}

\begin{itemize}
\tightlist
\item
  How does anxiety influence reaction time to fearful faces?
\item
  Does attentional control modify this relationship?
\item
  Are the findings specific to fearful stimuli, or would they generalize
  to other emotional expressions?
\end{itemize}

\begin{quote}
Review your code and interpretations. Cross-reference with theoretical
models on anxiety and attentional control.
\end{quote}

\chapter*{Dataskillsintro}\label{dataskillsintro}
\addcontentsline{toc}{chapter}{Dataskillsintro}

\markboth{Dataskillsintro}{Dataskillsintro}

\chapter{WhatIsScience (Week Link)}\label{whatisscience-week-link}

Please read \href{../textbook/01-WhatIsScience.qmd}{WhatIsScience}

\chapter{WhyStatistics (Week Link)}\label{whystatistics-week-link}

Please read \href{../textbook/01-WhyStatistics.qmd}{WhyStatistics}




\end{document}
